\section*{Is Side-Chain Entropy Important?}
\label{sec:is_side_chain_entropy_important_}
What can we conclude about the importance of side-chain entropy in determining the folded structure of a protein? From investigations on existing protein sequences and structures, it would seem that side-chain entropy has a role to play in the protein folding process, but before coming to a conclusion, we need to consider a few caveats. Hyperthermophiles do not exactly represent the average case when it comes to living systems. These heat-loving organisms are known to adopt a number of special enhancements, such as enriched GC-content in their genomes, to survive in harsh conditions. Optimizing \emph{all} aspects of their proteins, including side-chain entropy, may be nothing more than an extreme case of evolution. Certainly, this points to the importance of side-chain entropy in designing hardy, robust proteins, and this is certain to be useful to biotechnology, but does this say anything about side-chain entropy in the average case?

Looking at existing protein structures, the picture is actually not as clear as we might have liked. Yes, including side-chain entropy in a energy function for discriminating decoys from native structures improved the success rate, but does this say something about the importance of side-chain entropy in forming the native structure or does it say something about the inability of the decoy methods to produce realistic side-chain entropy values? For a fixed radius of gyration, there is remarkable variation in the amount of side-chain entropy observed. Even comparing X-ray and NMR structures of the same proteins reveals a good amount of variation in side-chain entropy. Certainly some of this variation is artifact of the methods used for structure determination, but does that leave any amount of side-chain entropy which is actually important for forming the native structure?

Approaching the question from the opposite direction, by attempting to repeat nature's grand experiment of designing proteins through alterations to amino acid sequences, it would seem we come up with contradictory results. Depending on the metric used, amino acid side-chains appear to be either uncorrelated or weakly correlated Either way, including side-chain entropy in the design process does not seem to get us any closer to reaching the same conclusions that nature has. Inclusion of side-chain entropy in design experiments is not without consequence, though. Doing so affects the amino acids we'd expect to be affected, those with the longest side-chains, by shifting their environmental bias toward the surface of protein structures. Side-chain entropy, included by means of free energy calculations, also seems to have a rather significant impact on which sequences are ultimately chosen as the finished design.

So which is it? Well, if we are being completely honest, there is no answer because we are asking the wrong question, or rather, our question is imperfect. There are some important aspects to proteins and the protein folding process that we must keep in mind when deciding the role that side-chain entropy should play. Regarding proteins themselves, we now understand that what is typically referred to as the ``native state'' is more like a ``native ensemble.'' Proteins do not reach a single, ground-state conformation at physiologically relevant temperatures. The structures that we observe through NMR and X-ray crystallography experiments represent ensemble averages. In reality, a folded protein is still a rather dynamic entity, and even if side-chain entropy is not vital in determining a protein fold, it will likely have a role to play in the dynamics.

Of course, we also need to be careful when referring to a ``protein fold.'' This term is used commonly in the literature in reference to two different but related concepts. There is the concept of a backbone fold defined by the trace of the peptide bonds joining a proteins amino acids. There is also the concept of a complete three dimensional structure including all backbone and side-chain atoms. There is almost no question that side-chain entropy has some sort of effect on the latter type of ``protein fold,'' but what of the former? Does side-chain entropy play a role in determining the arrangement of the backbone carbons in a protein?

Recently Rose, \emph{et al.} proposed a backbone-based theory of protein folding\cite{Rose:2006p362}. The reasons for developing such a theory include such observations as the limited number of folds possible given a remarkably large number of amino acid combinations and the relative unimportance of individual amino acids in maintaining the global fold of a protein. Specifically regarding this last observation, there is a rule of thumb used in the field of structural proteomics. It is accepted that two proteins which share 30\% sequence identity can be expected to have nearly identical backbone folds, and therefore are candidates for such methods as homology modeling and molecular replacement. Curiously, the amount of sequence identity between the designed proteins and their naturally occurring counterparts was also in the range of 30\%, regardless of whether side-chain entropy was accounted for. We might speculate, then, that the global backbone fold does not rely on side-chain entropy, but that there is a core requirement of residue interactions needed for a particular fold. Side-chain entropies may only come into play in determining the fine details of the protein's structure and function.