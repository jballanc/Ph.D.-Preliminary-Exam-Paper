\section*{How Is Side-Chain Entropy Important?}
\label{sec:how_is_side_chain_entropy_important_}
The question that we are really interested in, then, is in what ways is side-chain entropy important? It is a simple matter of physics that amino acid side-chains will have some amount of entropy, and that absent any additional forces this entropy will trend toward increasing. This is not being debated. What is currently unclear is which properties, if any, of a protein does this trend affect. There are a number of interesting experiments that remain to be done in pursuit of this knowledge, but I'd like to propose two which build on recent advances.

Galzitskaya and Garbuzynskiy recently carried out a series of experimental simulations indicating that there is an optimal relationship between conformational entropy of residues in a protein and the number of favorable contacts they maintain with other residues\cite{Galzitskaya:2006p35}. In particular, they suggest that this balance between entropy and contacts is requisite for proper protein function. I would propose taking this hypothesis combined with protein design techniques to create derivatives of well characterized proteins.

For such an experiment, one segment of sequence proximal to the active site and another segment at a greater distance would be subject to mutation using the \emph{de novo} design techniques described earlier, either including or excluding side-chain entropy in the calculations. The resulting sequences could then be synthesized and tested for activity. If side-chain entropy is required for function, then it would be expected that sequences designed with side-chain entropy considerations should be more likely to retain activity even if they do not retain native sequence. Further, by comparing the results for such an experiment carried out at two locations in the protein sequence, one might be able to determine if allosteric effects might be explained by side-chain entropy.

The second experiment I would like to see performed would involve pulling on hyperthermophilic bacterial proteins. Cao and Li were recently able to use atomic force microscopy equipment to pull on and denature a protein in the presence of a denaturing agent\cite{Cao:2008p142}. By doing so, they were able to determine that the denaturing agent reduced the energy barrier between folded and unfolded states. They were also able to determine that the transition state structure during unfolding was unaffected by the denaturing agent. It would be interesting to carry out these sorts of pulling experiments on proteins from hyperthermophilic bacteria both at elevated temperatures and in the presence of different denaturing agents.

Camilloni, \emph{et al.} showed that urea and guanidinium chloride act through distinct mechanisms to unfold proteins\cite{Camilloni:2008p276}. Specifically, guanadinium chloride preferably denatures $\alpha$-helices before $\beta$-sheets, and the reverse is true for urea. Thus, using one or the other in conjunction with atomic force microscope pulling at increasing temperatures, it should be possible to determine the entropy of these individual secondary structure units. While this is not side-chain entropy completely isolated by experiment, it is probably as close as we can come with current technology.