\section*{Introduction}
\label{sec:introduction}
Proteins are the work-horses of biology. They are responsible for the vast majority of the biochemical processes that take place in living organisms. It should come as no surprise, then, that there is keen interest in understanding how they function. Normally, this would be a question of brute-force experimentation, approaching and investigating proteins one at a time. Two aspects of proteins, however, make this a problem that is tantalizingly close, and yet still frustratingly far, from a complete solution.

The first useful aspect of proteins is the relationship between their structure and their function. Knowing a protein's structure goes a long way to understanding how it might function. This is particularly important when developing pharmaceutical agents to treat disease. Second, a protein's structure is completely determined by its amino acid sequence, which can in turn be determined from available genetic data. This was originally shown in 1961 by Anfinsen \cite{Service:2008p294} in his work with ribonuclease. This notion, of 3-dimensional structure data encoded in a 1-dimensional array of elements, has been verified for proteins in all but a few rare cases.

Thus, since knowing a protein's structure was already more than half the battle, and because it should be possible to determine that structure with readily available information, work began almost immediately on elucidating the mechanism drives proteins to form their unique structures. This work has been ongoing for the 47-years since.

Shortly after Anfinsen's discovery, Brandts proposed a two-state model for protein folding where a protein exists in an equilibrium, $\mathbf{\mathrm{D}} \rightleftarrows \mathbf{\mathrm{N}}$, between the denatured and native state\cite{Clark:2008p141}. This two-state model of protein folding is still the prevailing accepted model for folding, and it has some interesting implications. One of the first implications was recognized by Levinthal in 1968, and has since been known as Levinthal's paradox.

Essentially, what Levinthal realized was that Brandts' two-state model implied an absence of long-lived or marginally stable conformations. In other words, the denatured protein, which is essentially a random coil, must find the single lowest energy conformation from all of the possible conformations in one step. Levinthal's rough calculations implied that a robust search of all possible conformations would require more than the age of the universe to complete yet, paradoxically, proteins fold on a millisecond to second time scale. Obviously, proteins do not randomly sample conformation space. Instead, there must be a driving force that guides a denatured protein, no matter its conformation, toward the native state.

This understanding of Levinthal's Paradox lead Dill to propose the concept of the folding energy landscape as a funnel\cite{Dill:2008p283}, which has become an iconic symbol of the protein folding problem. The funnel captures the essence of the relationship between energy and conformational space. At high enough energies, proteins are free to explore all conformations, and are therefore ``denatured'' (in fact, the ``denatured'' state allows for some fraction of protein molecules to explore a native or native-like conformation, but this fraction is small enough to be inconsequential). As the energy of the protein is lowered, molecules with conformations near the center of the funnel will simply decrease in energy. However, those near the sides of the funnel will be steered toward more and more native-like conformations. In this way, Levinthal's paradox is resolved since all the molecules of a protein, regardless of their starting conformation, will converge on a point in conformation space as the energy is lowered.

Dill's funnel also captures another important aspect of protein folding: entropy. At any given energy, the volume of conformation space which remains inside the funnel corresponds to the conformational entropy of the protein. As a protein works its way down the ever narrower funnel, its entropy is decreasing. In order to offset this effect, the folded state of a protein must contain a host of favorable interactions. These will be the same sorts of interactions which define the funnel in the first place, and thus the problems of determining protein folding pathways, protein unfolding pathways, and native structures are all linked. 

Ultimately, the question of entropy's effect on protein folding is a rather complicated one. Entropy is, unlike enthalpy or potential energy, an ensemble property. That is, it cannot be determined by looking at a single folded molecule, but rather depends on how many different folds are possible for a given molecule at a given energy\cite{Meirovitch:2007p317}. This poses a significant problem in the calculation of entropy from molecular dynamics experiments. An exact calculation of the entropy of folding would require not only enumerating all of the possible native-like conformations accessible to a folded protein, but also to enumerating over all of the possible denatured protein conformations. While the former is merely computationally intensive, the later is intractable.

Entropy is also not very easy to separate in the same manner as enthalpy or potential\cite{Brady:1997p318}. Where two hydrogen bonds in a protein would be expected to have twice the stability of a single hydrogen bond, the same cannot be said, in general, for the entropy of two side-chains or a stretch of backbone dihedral angles. This makes it difficult to attack the problem of side-chain entropy with a piecemeal approach without making certain assumptions. This also makes difficult the task of directly measuring the contribution of one component of a protein to the protein's entropy.
