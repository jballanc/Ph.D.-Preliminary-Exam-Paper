\section*{Introduction}
\label{sec:introduction}
Proteins are the work-horses of biology. They are responsible for the vast majority of the biochemical processes that take place in living organisms. It should come as no surprise, then, that there is keen interest in understanding how they function. Normally, this would be a question of brute-force experimentation, approaching and investigating proteins one at a time. Two aspects of proteins, however, make this a problem that is tantalizingly close, and yet still frustratingly far, from a complete solution.

The first useful aspect of proteins is the relationship between their structure and their function. Knowing a protein's structure goes a long way to understanding how it might function. This is particularly important when developing pharmaceutical agents to treat disease. Second, a protein's structure is completely determined by its amino acid sequence, which can in turn be determined from available genetic data. This was originally shown in 1961 by Anfinsen \cite{Service:2008p294} in his work with ribonuclease. This notion, of 3-dimensional structure data encoded in a 1-dimensional array of elements, has been verified for proteins in all but a few rare cases.

Thus, since knowing a protein's structure was already more than half the battle, and because it should be possible to determine that structure with readily available information, work began almost immediately on elucidating the mechanism drives proteins to form their unique structures. This work has been ongoing for the 47-years since.

Shortly after Anfinsen's discovery, Brandts proposed a two-state model for protein folding where a protein exists in an equilibrium, $\mathbf{\mathrm{D}} \rightleftarrows \mathbf{\mathrm{N}}$, between the denatured and native state\cite{Clark:2008p141}. This two-state model of protein folding is still the prevailing accepted model for folding, and it has some interesting implications. One of the first implications was recognized by Levinthal in 1968, and has since been known as Levinthal's paradox.

Essentially, what Levinthal realized was that Brandts' two-state model implied an absence of long-lived or marginally stable conformations. In other words, the denatured protein, which is essentially a random coil, must find the single lowest energy conformation from all of the possible conformations in one step. Levinthal's rough calculations implied that a robust search of all possible conformations would require more than the age of the universe to complete yet, paradoxically, proteins fold on a millisecond to second time scale. Obviously, proteins do not randomly sample conformation space. Instead, there must be a driving force that guides a denatured protein, no matter its conformation, toward the native state.

This understanding of Levinthal's Paradox lead Dill to propose the concept of a folding funnel\cite{Dill:2008p283}, which has become an iconic symbol of the protein folding problem.

\begin{itemize}
	\item Dill's Funnel
	\item Difficulty in calculating Entropy from MD/X-ray structure
\end{itemize}
