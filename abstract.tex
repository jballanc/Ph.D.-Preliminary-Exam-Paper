Protein folding is a process which involves thermodynamic contributions from forces which favor the folded state and other forces which favor the unfolded state. When the balance of these forces tips in favor of the folded state, a protein will fold spontaneously into a unique conformation representing a thermodynamic minimum. The difficulty in predicting just how, and when, a protein will fold stems from a fundamental lack of knowledge about the forces involved. This is especially true in the case of the conformational entropy of amino acid side-chains (side-chain entropy).

Because of the difficulty associated with either directly measuring side-chain entropy or calculating its value from molecular simulations, there is uncertainty as to the effect it has on protein folding. The side-chains of amino acids range in size and flexibility, and some of the longer side-chains contain polar or even charged groups. It’s been proposed that side-chains might counteract a loss of entropy by a gaining interactions with other amino acids. Additionally, the hydrophobic residues commonly found in the core of a folded protein are not the most flexible, and so they will not be affected as significantly by the corresponding space constraints.

On the other hand, buried residues will loose some degree of side-chain entropy. Even though the cores of proteins rarely represent a closest-packed configuration, the side chains of long, charged residues found in internal salt bridges will undoubtedly be restricted in their motion. Furthermore, it is now understood that what is commonly refered to as the ``native protein structure'' is actually an ensemble of closely related structures. It is possible that side-chain entropy may not be important in determining the global fold of a protein, but may be vital in understanding the dynamics associated with function.

In this paper I will look at a variety of experiments which approach the question of side-chain entropy from differing perspectives. The first series of approaches involves investigating the structure and sequence of natural proteins, specifically focusing on side-chain entropy. The alternative approach is to perform \emph{de novo} protein design either using or neglecting side-chain entropy in the calculations. Both of these approaches are, necessarily, indirect, and each leads to a different conclusion. I will look at the results in context, and discuss whether or not a real controversy exists. Finally, I will propose a course of investigation which might shed more light on the role of side-chain entropy in protein folding.

\begin{description}
	\item[Topics] 
	\item Thermodynamics 
	\item Protein Folding
	\item Statistical Mechanics
	\item Molecular Dynamics
	\item X-ray Crystallography
	\item \emph{de novo} Protein Design
\end{description}