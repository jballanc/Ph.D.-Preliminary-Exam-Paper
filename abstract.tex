Protein folding is a process which involves thermodynamic contributions from forces which favor the folded state and other forces which favor the unfolded state. When the balance of these forces tips in favor of the folded state, a protein will spontaneously fold into a unique conformation which represents a global thermodynamic minimum. The difficulty in predicting just how, and when, a protein will fold stems from a fundamental lack of knowledge about the forces involved. Traditionally, it has been held that the forces in favor of folding are Hydrogen-bonding, salt-bridging, and hydrophobic and van der Waals interactions; while the primary force opposing folding is the decrease in backbone and side-chain entropies.

Because of the difficulty associated with either directly measuring side-chain entropy or calculating its from molecular simulations, there is uncertainty not only in the amount of effect on protein folding, but also in whether it has an appreciable effect at all. The side-chains of amino acids range in size and flexibility, and some of the longer side-chains contain polar or even charged groups. It’s been proposed that these groups can counteract a loss of side-chain entropy by a gain in interactions with other residues. Additionally, the hydrophobic residues which are commonly found in the core of a folded protein structure are not the most flexible.

On the other hand, that buried residues will loose some degree of side-chain entropy is not in doubt. Even though the cores of proteins rarely represent a closest-packed configuration, the side chains of long, charged residues often found in internal salt bridges, such as arginine or glutamate, will undoubtedly be restricted in their motion. Furthermore, it is now understood that what is commonly refered to as the ``native protein structure'' is actually an ensemble of closely related structures. It is possible that side-chain entropy may not be important in determining the global structure of a protein, but may be vital in understanding the dynamics associated with catalysis.

In this paper I will look at two sets of papers which approach the question of side-chain entropy from differing perspectives. First, one way to evaluate the importance of side-chain entropy in determining the structure of a protein is to look at the experimentally determined structure of a collection of proteins. By calculating the side-chain entropy of these existing structures, one should be able to discern the relative importance. Indeed, one group has found that accounting for side-chain entropy significantly improves their ability to separate real structures from corresponding decoys. A second approach to the same problem is to design proteins \emph{de novo}. In the design process, one can choose which forces to use for evaluating candidate structures. In just such an experiment, one group has found that their results are not substantially altered by the inclusion of side-chain entropy in the design algorithm.

\begin{description}
	\item[Topics] 
	\item Thermodynamics 
	\item Protein Folding
	\item Statistical Mechanics
	\item Molecular Dynamics
	\item X-ray Crystallography
	\item \emph{de novo} Protein Design
\end{description}